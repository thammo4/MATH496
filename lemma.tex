 \documentclass{article}
 \usepackage{amsmath, amssymb, xpatch, amsthm, outlines, eucal}
\usepackage[utf8]{inputenc}
\usepackage[shortlabels]{enumitem}


\usepackage{mathrsfs}
\usepackage{multicol}
\usepackage{bm}







\theoremstyle{plain}

\makeatletter
\xpatchcmd{\@thm}{\thm@headpunct{.}}{}{}


\newtheorem*{math496*}{Kevin's Lemma}





\newcommand{\V}[1]{\textbf{#1}}

%\begin{equation*}
%F \left(x\right) := \left\{
%	\begin{array}{ll}
%		\int\limits_{a}^{x} f\left(t\right) dt  & ; \quad x \in \left[a,b\right] \vspace{.15 em} \\
%		0 & ; \quad \text{\rm{else}}
%	\end{array}
%\right.
%\end{equation*}





\begin{document}

\begin{math496*}

\noindent Let $ f \in L^{2} \left(\mathbb{R}\right) $. Suppose $ V_{\epsilon} \left(x\right) \in L^{\infty} \left(\mathbb{R}\right) $, where $ \epsilon \in \left(0,1\right] $ is arbitrary. Assume as $ \epsilon \to 0 $, $ || V_{\epsilon} ||_{L^{2}} \to 0 $. It then follows that as $ \epsilon \to 0 $, we have:
$$ || V_{\epsilon} \cdot f ||_{L^{2}} \to 0 $$

\end{math496*}


\begin{proof} The $ L^{2} $ function space respects the triangle inequality, so we can rewrite $ || V_{\epsilon} \left(x\right) f\left(x\right) ||_{L^{2}} $ as:
\begin{align*}
|| V_{\epsilon} f ||_{L^{2}} & = || V_{\epsilon} f + \underbrace{V_{\epsilon}f_{\delta}-V_{\epsilon}f_{\delta}}_{=0}||_{L^{2}} \\
& = || V_{\epsilon}f_{\delta} + V_{\epsilon}f - V_{\epsilon}f_{\delta} ||_{L^{2}} \\
& = || V_{\epsilon}f_{\delta} + V_{\epsilon}\left(f-f_{\delta}\right)||_{L^{2}} \\
& \leq || V_{\epsilon}f_{\delta}||_{L^{2}} + ||V_{\epsilon}\left(f-f_{\delta}\right)||_{L^{2}} \\
& \leq || V_{\epsilon} f_{\delta} ||_{L^{2}} + ||V_{\epsilon}||_{L^{\infty}} \cdot ||f-f_{\delta}|| _{L^{2}}
\end{align*}
We will demonstrate that as $ \epsilon \in \mathbb{R}^{+}  \to 0 $, the expression on the right side of the inequality, $  || V_{\epsilon} f_{\delta} || + || V_{\epsilon} || \cdot || f - f_{\delta} || $, will go to zero. \\

\noindent Consider the first term of the sum, $ || V_{\epsilon} f_{\delta} || = || V_{\epsilon} || \cdot || f_{\delta} || $. By assumption, we know  as $ \epsilon \to 0 $, that $ ||V_{\epsilon}|| \to 0 $. Moreover, since $ f_{\delta} \in L^{2} $, we have that $ \text{\rm{sup}} | f_{\delta} | $ is bounded, and $ || f_{\delta} || \leq \text{\rm{sup}}|f_{\delta}| $, so:
\begin{align*}
|| V_{\epsilon} f_{\delta} || & = || V_{\epsilon} || \cdot || f_{\delta} || \\
& \leq || V_{\epsilon} || \cdot \left( \text{\rm{sup}} |f_{\delta}| \right) \\
& = 0 \cdot \left( \text{\rm{sup}}|f_{\delta}|\right) \\
& = 0
\end{align*}

\noindent Now observe that $ || V_{\epsilon} \left( f - f_{\delta}\right) || = || V_{\epsilon} || \cdot || f - f_{\delta} || $. Since $ V_{\epsilon} \in L^{\infty} $, it is an essentially bounded function, and this means that it is bounded except for a set $ \left\{ x \hspace{.25 em} : \hspace{.25 em} x \in \mathbb{R} \right\} $ having measure zero. Notably, we have that $ \text{\rm{sup}} |V_{\epsilon}| < \infty $. Recall that our function $ f_{\delta} \in L^{2} $ is a continuous, compactly supported function such that for an arbitraryt $ \delta > 0 $,
$$  || f - f_{\delta} || \leq \delta $$
If we let $ \delta \to 0 $, then $ || f - f_{\delta} || \to 0 $. Thus we have:
\begin{align*}
|| V_{\epsilon} \left(f-f_{\delta}\right) || & = || V_{\epsilon} || \cdot || f - f_{\delta} || \\
& \leq \left(\text{\rm{sup}}|V_{\epsilon}|\right) \cdot ||f-f_{\delta}|| \\
& = \left(\text{\rm{sup}}|V_{\epsilon}|\right) \cdot 0 \\
& = 0 \quad;\hspace{.2 em} \left(\delta \to 0\right)
\end{align*}
We've now shown that as $ \epsilon \to 0 $, $ || V_{\epsilon} f_{\delta} || \to 0 $ and $ || V_{\epsilon} \left(f - f_{\delta} \right) || \to 0 $. Having already established that for arbitrary $ \epsilon > 0$,  $ || V_{\epsilon} f || \leq ||V_{\epsilon} f_{\delta} || + || V_{\epsilon} \left(f-f_{\delta}\right)|| $, we conclude that:
$$ || V_{\epsilon} f ||_{L^{2}} \to 0 $$
\end{proof}





%$$ || V_{\epsilon} f ||_{L^{2}} = || V_{\epsilon} f_{\delta} + V_{\epsilon} \left(f - f_{\delta}\right) ||_{L^{2}} \leq || V_{\epsilon} f_{\delta} || + || V_{\epsilon} \left(f - f_{\delta}\right) || $$
%For the first term on the right side of the inequality, $ || V_{\epsilon} f_{\delta} || $, we have that:
%$$ || V_{\epsilon} f_{\delta} || \leq \left( \text{\rm{sup}} |V_{\epsilon}|\right) \cdot || V_{\epsilon} || $$
%And because $ \lim\limits_{\epsilon \to 0} ||V_{\epsilon}||_{L^{2}} = 0 $, we know that:
%$$ \left( \text{\rm{sup}}|V_{\epsilon}|\right) \cdot || V_{\epsilon} || = \left(\text{\rm{sup}}|V_{\epsilon}|\right) \cdot 0 = 0 $$
%
%\noindent Since $ || V_{\epsilon} f_{\delta} || \leq \left( \text{\rm{sup}}|V_{\epsilon}| \right) \cdot || V_{\epsilon} || $ and $ \left( \text{\rm{sup}}|V_{\epsilon}| \right) \cdot || V_{\epsilon} || = 0 $, it follows that:
%\begin{align*}
%\lim\limits_{\epsilon \to 0} || V_{\epsilon} f_{\delta} || & \leq \lim\limits_{\epsilon \to 0} \left( \text{\rm{sup}} |V_{\epsilon}| \right) \cdot ||V_{\epsilon} || \\
%& \leq \left( \lim\limits_{\epsilon \to 0} \text{\rm{sup}}|V_{\epsilon}|  \right) \cdot \left( \lim\limits_{\epsilon \to 0} ||V_{\epsilon}||\right) \\
%& = \left( \lim\limits_{\epsilon \to 0} \text{\rm{sup}} |V_{\epsilon} | \right) \cdot 0 \\
%& = 0
%\end{align*}
%$ L^{p} $ norms are always nonnegative, so:
%\begin{align*}
%& 0 \leq \lim\limits_{\epsilon \to 0} || V_{\epsilon} f_{\delta}|| \leq 0 \\
%\Leftrightarrow & \lim\limits_{\epsilon \to 0} ||V_{\epsilon} f_{\delta}|| = 0
%\end{align*}
\end{document}
%
%
%
%
%
%
%
%
%
%
%
%
%
%
%
%
%
%
%
%
%
%
%
%
%
%
%
%\end{document}
